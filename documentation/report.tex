%-------------------------------------------------------------------------------
%	NAME:	report.tex
%	AUTHOR: Connor Beardsmore - 15504319
%	LAST MOD:	28/09/17
%	PURPOSE:	FCC Assignment Report
%	REQUIRES:	NONE
%-------------------------------------------------------------------------------

\documentclass[]{article}
\usepackage[ margin=3cm ]{geometry}
\usepackage{graphicx}
\usepackage{fancyhdr}
\usepackage{float}
\usepackage{hyperref}
\usepackage{transparent}
\usepackage{multicol}
\usepackage{amsmath}
\usepackage[final]{pdfpages}
\usepackage{listings}
\usepackage{color}
\usepackage{algorithmicx}
\usepackage{algpseudocode}
\usepackage{amssymb}

\pagestyle{fancy}
\fancyhf{}
\lhead{Connor Beardsmore - 15504319}
\rhead{FCC200}
\lfoot{April 2017}
\rfoot{\thepage}

\pagenumbering{arabic}
\graphicspath{{./images/}}

%-------------------------------------------------------------------------------
% CODE HIGHLIGHTING FOR LISTINGS
\definecolor{codegreen}{rgb}{0,0.6,0}
\definecolor{codegray}{rgb}{0.5,0.5,0.5}
\definecolor{codepurple}{rgb}{0.58,0,0.82}
\definecolor{backcolour}{rgb}{0.99,0.99,0.99}

\lstdefinestyle{mystyle}{
	backgroundcolor=\color{backcolour},   
	commentstyle=\color{codegreen},
	keywordstyle=\color{magenta},
	numberstyle=\tiny\color{codegray},
	stringstyle=\color{codepurple},
	basicstyle=\footnotesize,
	breakatwhitespace=false,         
	breaklines=true,                 
	captionpos=b,                    
	keepspaces=true,                 
	numbers=left,                    
	numbersep=5pt,                  
	showspaces=false,                
	showstringspaces=false,
	showtabs=false,                  
	tabsize=2
}

\lstset{style=mystyle}


%-------------------------------------------------------------------------------
\begin{document}
%-------------------------------------------------------------------------------
% TITLE PAGE

\begin{titlepage}
	\begin{center}
		\vspace*{1cm}
		\LARGE\textbf{FCC200 Report}
		\break
		Affine Cipher and S-DES Implementation
		\vspace{1cm}
		\break
		\Large\textbf{Connor Beardsmore - 15504319} 
		\vspace{15cm}

		\normalsize
		Curtin University \\
		Science and Engineering \\
		Perth, Australia \\
	    April 2017
	    
	\end{center}
\end{titlepage}

%-------------------------------------------------------------------------------
% AFFINE CIPHER

\vspace*{-0.8cm}
\begin{center}
	\section*{Affine Cipher}
\end{center}

\vspace*{0.8cm}
\subsection*{Compute Eligible Keys}

There are two keys required, \textit{a} and \textit{b}. The first is required to be \textit{coprime} with the length of the alphabet, in this scenario \textit{26}. The second key representing the linear shift must be both positive and less than the length of the alphabet. To check if the \textit{a} value is coprime, the following greatest common denominator check was utilized. If the greatest common denominator of \textit{a} and 26 is 1, the value of \textit{a} is \textit{coprime} and the key is valid in combination with any valid \textit{b} value.\\

\lstinputlisting[language=C,linerange={34-60} ]{../affine/keyeligible.c} 

\vspace{0.5cm}
The results of calling this function on all \textit{a} values from 1 to 25 are as follows:\\

\begin{multicols}{2}
	\begin{itemize}
		\item \textbf{gcdFunction( 1, 26 ) = 1}
		\item gcdFunction( 2, 26 ) = 2
		\item \textbf{gcdFunction( 3, 26 ) = 1}
		\item gcdFunction( 4, 26 ) = 2
		\item \textbf{gcdFunction( 5, 26 ) = 1}
		\item gcdFunction( 6, 26 ) = 2
		\item \textbf{gcdFunction( 7, 26 ) = 1}
		\item gcdFunction( 8, 26 ) = 2
	    \item \textbf{gcdFunction( 9, 26 ) = 1}
		\item gcdFunction( 10, 26 ) = 2
		\item \textbf{gcdFunction( 11, 26 ) = 1}
		\item gcdFunction( 12, 26 ) = 2
		\item gcdFunction( 13, 26 ) = 13
		\item gcdFunction( 14, 26 ) = 2
		\item \textbf{gcdFunction( 15, 26 ) = 1}
		\item gcdFunction( 16, 26 ) = 2
		\item \textbf{gcdFunction( 17, 26 ) = 1}
		\item gcdFunction( 18, 26 ) = 2
		\item \textbf{gcdFunction( 19, 26 ) = 1}
		\item gcdFunction( 20, 26 ) = 2
		\item \textbf{gcdFunction( 21, 26 ) = 1}
		\item gcdFunction( 22, 26 ) = 2	
		\item \textbf{gcdFunction( 23, 26 ) = 1}
		\item gcdFunction( 24, 26 ) = 2
		\item \textbf{gcdFunction( 25, 26 ) = 1}
	\end{itemize}
\end{multicols}


\vspace{0.5cm}

The full list of valid \textit{a} values is: \textbf{1, 3, 5, 7, 9, 11, 15, 17, 19, 21, 23, 25}\\

There are a total of 12 possible \textit{a} values that are coprime with \textit{26}. Each of these values can have a shift value (\textit{b}) of 0 to 25. Thus, the total number of eligible keys is:

$$12*26=312$$

Of these, 26 keys are trivial Caesar ciphers and 286 are non-trivial.

\subsection*{Recovered Plaintext}

\begin{figure}[H]
	\includegraphics[width=\textwidth]{affine_plaintext.png}
	\caption{Original Plaintext}
	\centering
\end{figure}

\begin{figure}[H]
	\includegraphics[width=\textwidth]{affine_encrypt.png}
	\caption{Encryption Process}
	\centering
\end{figure}

\begin{figure}[H]
	\includegraphics[width=\textwidth]{affine_ciphertext.png}
	\caption{Encrypted Ciphertext}
	\centering
\end{figure}

\begin{figure}[H]
	\includegraphics[width=\textwidth]{affine_decrypt.png}
	\caption{Decryption Process}
	\centering
\end{figure}

\begin{figure}[H]
d	\includegraphics[width=\textwidth]{affine_plaintext.png}
	\caption{Recovered Plaintext}
	\centering
\end{figure}

\subsection*{Affine Mathematical Proof}

The encryption and decryption functions for the affine cipher are as follows:

$$E(x)=(ax+b)\;mod\;m$$

$$D(x)=a^{-1}(x-b)\;mod\;m$$

\vspace{0.5cm}

The modular multiplicative inverse of \textit{a} is defined as:

$$1=aa^{-1}\;mod\;m$$

It can be shown that \textit{D(x)} is the inverse of \textit{E(x)} via the modular arithmetic laws.

\begin{equation*}
\begin{split}
D(E(x)) & = a^{-1}(E(x)-b)\;mod\;m \\
& = a^{-1}(( ax+b\;mod\;m )-b)\;mod\;m \\
& = a^{-1} (ax+b-b) \;mod\;m \\
& = a^{-1}ax\;mod\;m \\
& = x\;mod\;m
\end{split}
\end{equation*}




\subsection*{Letter Distribution}

For the given test file shown in Figure 2, the following table and Figure 7 illustrate the letter distribution and relative frequencies.

\begin{multicols}{7}
	\begin{itemize}
		\item A: 35
		\item B: 7
		\item C: 17
		\item D: 14
		\item E: 65
		\item F: 14
		\item G: 10
		\item H: 16
		\item I: 41
		\item J: 0
		\item K: 0
		\item L: 18
		\item M: 16
		\item N: 35
		\item O: 50
		\item P: 23
		\item Q: 2
		\item R: 39
		\item S: 39
		\item T: 53
		\item U: 8
		\item V: 2
		\item W: 4
		\item X: 2
		\item Y: 3
		\item Z: 1
	\end{itemize}
\end{multicols}

\begin{figure}[H]
	\includegraphics[width=\textwidth]{frequency.png}
	\caption{Letter Distributions}
	\centering
\end{figure}

\break
%-------------------------------------------------------------------------------
% S-DES ENCRYPTION

\vspace*{-0.8cm}
\begin{center}
	\section*{S-DES}
\end{center}

\vspace*{0.8cm}
\subsection*{S-DES Mathematical Proof}

hello

\subsection*{Pseudo Code Structure}

The pseudo-code structure of the three key functions utilized in the S-DES implementation is illustrated below.\\

\begin{algorithmic}
\Function{KeyGeneration}{int key}
	\State $key\gets $ \Call{permute}{ key, P10 }
	\State \Call{leftshift}{ key, 1 }
	\State $subkeys[0]\gets $ \Call{permute}{ key, P8 }
	\State \Call{leftshift}{ key, 2 }
	\State $subkeys[1]\gets $ \Call{permute}{ key, P8 }
	\State \Return subkeys
\EndFunction
\end{algorithmic}


\vspace{0.5cm}

\begin{algorithmic}
	\Function{SwitchFunction}{int input}
		\State $ right \gets bits \;\&\&\; ( ( 1 << 4 ) - 1 ) $
		\State $ left \gets bits >>> 4 $
		\State $ output \gets left \;||\; ( right << 4 )$
		\State \Return output
	\EndFunction
\end{algorithmic}

\vspace{0.5cm}

\begin{algorithmic}
	\Function{FeistalKeyRound}{int message, int subkey}
	
	\State $halves\gets $ \Call{split}{ message }
	\State $fMap \gets $ \Call{fMapping}{ rightHalf, subkey }	
	\State $ leftHalf \gets leftHalf \oplus fMap $
	\State $ combined \gets leftHalf + rightHalf $
	\State \Return combined
	\EndFunction
\end{algorithmic}

\subsection*{Encrypted Test File}

\begin{figure}[H]
	\includegraphics[width=\textwidth]{sdes_encrypt.png}
	\caption{S-DES Encryption Process}
	\centering
\end{figure}

\begin{figure}[H]
	\includegraphics[width=\textwidth]{sdes_cipher1.png}
	\includegraphics[width=\textwidth]{sdes_cipher2.png}	
	\caption{S-DES Cipher Text}
	\centering
\end{figure}

\subsection*{Decrypted Test File}

\begin{figure}[H]
	\includegraphics[width=\textwidth]{sdes_decrypt.png}
	\caption{S-DES Decryption Process}
	\centering
\end{figure}

\begin{figure}[H]
	\includegraphics[width=\textwidth]{sdes_plain1.png}
	\includegraphics[width=\textwidth]{sdes_plain2.png}	
	\caption{S-DES Plain Text}
	\centering
\end{figure}

\subsection*{Utilization of an all 1 Key}

Performing encryption and decryption with S-DES utilizing a key of all 1's (11111111) does not significantly alter how the algorithm performs. However, during the two feistal key rounds, the subkeys will be equivalent. This leads to both encryption and decryption being the same process. Thus in this situation:

$$x=E( E(x) )$$

\subsection*{Modify S-Boxes}


The S-BOX values in the SDESConstants.java file were modified to ensure that the S-DES algorithm still performs accurately.

\break
%-------------------------------------------------------------------------------
% FINAL QUESTIONS

\vspace*{-0.8cm}
\begin{center}
	\section*{Additional Questions}
\end{center}

\vspace*{0.8cm}
\subsection*{Threats}

hello \\

\subsection*{Source Coding}

Source coding in information transmission aims to compress natural messages for highly efficient message transfer.

\subsection*{Error Coding}

Error coding in information transmission attempts to enable a high information rate by the introduction of redundancy to data, as well as via error detection and correction mechanisms.

\subsection*{S-DES Coding}

hello \\

\subsection*{S-DES Confusion and Diffusion}

In S-DES, confusion is provided by the S-BOX substitutions performed within the feistal key round. Diffusion in contrast is provided by the permutations applied to the data included the expansion permutation utilized.

\pagebreak

%-------------------------------------------------------------------------------
% AFFINE CODE

\vspace*{-0.8cm}
\begin{center}
	\section*{Affine Source Code}
\end{center}

\subsection*{keyeligible.h}
\lstinputlisting[language=C,linerange={} ]{../affine/keyeligible.h}\pagebreak{}
\subsection*{keyeligible.c}
\lstinputlisting[language=C,linerange={} ]{../affine/keyeligible.c}\pagebreak{}
\subsection*{affine.h}
\lstinputlisting[language=C,linerange={} ]{../affine/affine.h}\pagebreak{}
\subsection*{affine.c}
\lstinputlisting[language=C,linerange={} ]{../affine/affine.c}\pagebreak{}

%-------------------------------------------------------------------------------
% SDES CODE

\vspace*{-0.8cm}
\begin{center}
	\section*{S-DES Source Code}
 \end{center}

\subsection*{SDESConstants.java}
\lstinputlisting[language=java,linerange={} ]{../SDES/SDESConstants.java}\pagebreak{}
\subsection*{SDESBits.java}
\lstinputlisting[language=java,linerange={} ]{../SDES/SDESBits.java}\pagebreak{}
\subsection*{SDES.java}
\lstinputlisting[language=java,linerange={} ]{../SDES/SDES.java}\pagebreak{}

%-------------------------------------------------------------------------------
\end{document}   
%-------------------------------------------------------------------------------: