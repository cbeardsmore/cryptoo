%-------------------------------------------------------------------------------
%	NAME:	report.tex
%	AUTHOR: Connor Beardsmore - 15504319
%	LAST MOD:	01/05/17
%	PURPOSE:	FCC Assignment Report
%	REQUIRES:	NONE
%-------------------------------------------------------------------------------

\documentclass[]{article}
\usepackage[ margin=3cm ]{geometry}
\usepackage{graphicx}
\usepackage{fancyhdr}
\usepackage{float}
\usepackage{hyperref}
\usepackage{transparent}
\usepackage{multicol}
\usepackage{amsmath}
\usepackage[final]{pdfpages}
\usepackage{listings}
\usepackage{color}
\usepackage{algorithmicx}
\usepackage{algpseudocode}
\usepackage{amssymb}
\usepackage[style=chicago-authordate,backend=biber]{biblatex}

\pagestyle{fancy}
\fancyhf{}
\lhead{Connor Beardsmore - 15504319}
\rhead{FCC200}
\lfoot{May 2017}
\rfoot{\thepage}

\pagenumbering{arabic}
\graphicspath{{./images/}}

\addbibresource{bib/references.bib}
\nocite{*}

%-------------------------------------------------------------------------------
% CODE HIGHLIGHTING FOR LISTINGS
\definecolor{codegreen}{rgb}{0,0.6,0}
\definecolor{codegray}{rgb}{0.5,0.5,0.5}
\definecolor{codepurple}{rgb}{0.58,0,0.82}
\definecolor{backcolour}{rgb}{0.99,0.99,0.99}

\lstdefinestyle{mystyle}{
	backgroundcolor=\color{backcolour},   
	commentstyle=\color{codegreen},
	keywordstyle=\color{magenta},
	numberstyle=\tiny\color{codegray},
	stringstyle=\color{codepurple},
	basicstyle=\footnotesize,
	breakatwhitespace=false,         
	breaklines=true,                 
	captionpos=b,                    
	keepspaces=true,                 
	numbers=left,                    
	numbersep=5pt,                  
	showspaces=false,                
	showstringspaces=false,
	showtabs=false,                  
	tabsize=2
}

\lstset{style=mystyle}


%-------------------------------------------------------------------------------
\begin{document}
%-------------------------------------------------------------------------------
% TITLE PAGE

\includepdf[]{./images/cover_page.pdf}

\begin{titlepage}
	\begin{center}
		\vspace*{1cm}
		\LARGE\textbf{FCC200 Report}
		\break
		RSA Cryptosystem Implementation
		\vspace{1cm}
		\break
		\Large\textbf{Connor Beardsmore - 15504319} 
		\vspace{15cm}

		\normalsize
		Curtin University \\
		Science and Engineering \\
		Perth, Australia \\
	    May 2017
	    
	\end{center}
\end{titlepage}

%-------------------------------------------------------------------------------
% BINARY MODULAR EXPONENTIATION
\vspace*{-0.8cm}
\section*{\hfil RSA Implementation\hfil}

\subsection*{Modular Exponentiation}
\noindent
Modular exponentiation is used to calculate the remainder when a base \textit{b} is raised to an exponent \textit{e} and reduced by some modulus \textit{m}. The simple right-to-left method provided by \cite{alttext} utilizes exponentiation by squaring. The full Java code for the implementation of this method is illustrated below. The running time of this algorithm is $O(log\;e)$ which provides a significant improvement over more simplistic methods of time complexity $O(e)$ (\cite{maintext}). This calculation is a core component of RSA and thus its efficiency is crucial to the speed of the RSA implementation.

\vspace{0.2cm}
\lstinputlisting[language=C,linerange={59-92} ]{../../rsa/numberTheory.c}{}

\noindent
The code above was utilized to calculate the following example:

$$236^{239721}\;mod\;2491=236$$

\vspace{0.2cm}
\noindent
The running of this code provided the following output:

\begin{figure}[H]
		\centering
	\includegraphics[height=\textheight/10,width=\textwidth/3]{exponentiation.png}
	\caption{Modular Exponentiation Example}
\end{figure}

\pagebreak

%-------------------------------------------------------------------------------
% RSA IMPLEMENTATION
\vspace*{-0.8cm}

\subsection*{RSA Testing}

The implemented RSA cipher works correctly with any key generated by the system. The recovered plaintext in Figure 4 is identical to the original plaintext of Figure 2, as confirmed by the Linux \textit{diff} command. No major problems were encountered during the implementation phase. The bytes visible in the ciphertext are not readable as plaintext and as a result, standard text editors cannot appropriately display the information.

\vspace{0.5cm}
\begin{figure}[H]
	\includegraphics[height=\textheight/6,width=\textwidth]{rsa_plain1.png}
	\includegraphics[height=\textheight/6,width=\textwidth]{rsa_plain2.png}	
	\caption{RSA Plaintext}
	\centering
\end{figure}

\begin{figure}[H]
	\includegraphics[width=\textwidth]{rsa_cipher1.png}
	\includegraphics[width=\textwidth]{rsa_cipher2.png}	
	\caption{RSA Ciphertext}
	\centering
\end{figure}

\begin{figure}[H]
	\includegraphics[height=\textheight/6,width=\textwidth]{rsa_plain1.png}
	\includegraphics[height=\textheight/6,width=\textwidth]{rsa_plain2.png}	
	\caption{RSA Recovered Plaintext}
	\centering
\end{figure}

\pagebreak

%-------------------------------------------------------------------------------
% ADDITIONAL QUESTIONS

\vspace*{-0.8cm}
\section*{\hfil Additional Questions\hfil}

\subsection*{Signature Forgery}

RSA can be utilized as a message signature scheme to provide authentication to messages. If Alice wants to send a signed message to Bob, she first produces a \textit{hash value} of the message $H(m)$, then raises this to the power $d(modulo\;n)$ and attaches it to the message as a signature. When Bob receives the message, he utilizes the same hashing function, raises the result to the power of $e(modulo\;n)$ and compares to the message signature ($H(m)=H(m')$). If they are the same, Bob can be assured that the message was signed by Alice or someone with knowledge of Alice's private key.\\

\begin{figure}[H]
	\centering
	\includegraphics[height=\textheight/7,width=12cm]{rsasignature.png}
	\caption{RSA Signature Scheme (\cite{alttext})}
\end{figure}

\noindent
In this scheme, it is not possible to completely ensure the message was sent by Alice. If Bob has some alternate message $m''$ with a hash value $H(m'')$ matching that of Alice's $H(m)$ as discussed in \cite{RSA}, Bob can pretend to be Alice. He can simply resend Alice's signature he received onwards and the receiver of this message will believe that the message was sent by Alice. He can then perform malicious actions such as replay attacks depending on his intent. This can only occur if Bob finds a hash collision where $H(m)=H(m')$. It is however unlikely for Bob to create a \textit{meaningful} message with a hash value matching Alice's (\cite{alttext}). This is analogous to the idea of Birthday attacks mentioned in \cite{lecture3}.\\

\noindent
To prevent this situation from occurring, the hash function utilized for the digital signature must be strongly collision resistant (\cite{RSA}). Strong collision resistance asserts that there exists no $m$ and $m'$ where $m!=m'$ so that $H(m)=H(m')$. Thus if the hash function adheres to this property, the above situation cannot occur.

\break
\subsection*{Birthday Attack}

\vspace{0.5cm}
\begin{center}
\textit{In a group of 23 randomly selected people, the probability that two\\ of them share the same birthday is larger than 50\%}
\end{center}
\vspace{0.5cm}

\noindent
Firstly, the probability that two people have different birthdays is found:

$$1-\frac{1}{365}=\frac{364}{365}=0.99726$$

\vspace{0.5cm}
\noindent
This can be extended to determine if three people have different birthdays:

$$1-\frac{2}{365}=\frac{363}{365}=0.99452$$

\vspace{0.5cm}
\noindent
Utilizing conditional probability (\cite{lecture2}) we can construct the probability that all 23 people have different birthdays. This is simply represented as a series of fractions with their product producing the resultant probability:

$$1\times(1-\frac{1}{365})(1-\frac{2}{365})...(1-\frac{22}{365})=0.493$$

\vspace{0.5cm}
\noindent
To find the probability that two of the people have the same birthday, we inverse this number by subtracting from the total probability (1):

$$1-0.493=0.507=50.7\%$$

\vspace{0.5cm}
\noindent
It is thus evident that the probability of two people in a set of 23 random selected sharing the same birthday is greater than 50\%.

\pagebreak

%-------------------------------------------------------------------------------
% RSA CODE

\vspace*{-0.8cm}
\begin{center}
	\section*{RSA Source Code}
\end{center}

\subsection*{makefile}
\lstinputlisting[language=make,linerange={} ]{../../rsa/Makefile}\pagebreak{}
\subsection*{numberTheory.h}
\lstinputlisting[language=C,linerange={} ]{../../rsa/numberTheory.h}\pagebreak{}
\subsection*{numberTheory.c}
\lstinputlisting[language=C,linerange={} ]{../../rsa/numberTheory.c}\pagebreak{}
\subsection*{main.h}
\lstinputlisting[language=C,linerange={} ]{../../rsa/main.h}\pagebreak{}
\subsection*{main.c}
\lstinputlisting[language=C,linerange={} ]{../../rsa/main.c}\pagebreak{}

%-------------------------------------------------------------------------------   
% REFERENCES

\break
\setlength\bibitemsep{4\itemsep}
\printbibliography[title={References}]

%-------------------------------------------------------------------------------
\end{document}   
%-------------------------------------------------------------------------------: